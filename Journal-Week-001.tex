% Options for packages loaded elsewhere
\PassOptionsToPackage{unicode}{hyperref}
\PassOptionsToPackage{hyphens}{url}
%
\documentclass[
]{article}
\usepackage{amsmath,amssymb}
\usepackage{lmodern}
\usepackage{ifxetex,ifluatex}
\ifnum 0\ifxetex 1\fi\ifluatex 1\fi=0 % if pdftex
  \usepackage[T1]{fontenc}
  \usepackage[utf8]{inputenc}
  \usepackage{textcomp} % provide euro and other symbols
\else % if luatex or xetex
  \usepackage{unicode-math}
  \defaultfontfeatures{Scale=MatchLowercase}
  \defaultfontfeatures[\rmfamily]{Ligatures=TeX,Scale=1}
\fi
% Use upquote if available, for straight quotes in verbatim environments
\IfFileExists{upquote.sty}{\usepackage{upquote}}{}
\IfFileExists{microtype.sty}{% use microtype if available
  \usepackage[]{microtype}
  \UseMicrotypeSet[protrusion]{basicmath} % disable protrusion for tt fonts
}{}
\makeatletter
\@ifundefined{KOMAClassName}{% if non-KOMA class
  \IfFileExists{parskip.sty}{%
    \usepackage{parskip}
  }{% else
    \setlength{\parindent}{0pt}
    \setlength{\parskip}{6pt plus 2pt minus 1pt}}
}{% if KOMA class
  \KOMAoptions{parskip=half}}
\makeatother
\usepackage{xcolor}
\IfFileExists{xurl.sty}{\usepackage{xurl}}{} % add URL line breaks if available
\IfFileExists{bookmark.sty}{\usepackage{bookmark}}{\usepackage{hyperref}}
\hypersetup{
  hidelinks,
  pdfcreator={LaTeX via pandoc}}
\urlstyle{same} % disable monospaced font for URLs
\usepackage[margin=1in]{geometry}
\usepackage{graphicx}
\makeatletter
\def\maxwidth{\ifdim\Gin@nat@width>\linewidth\linewidth\else\Gin@nat@width\fi}
\def\maxheight{\ifdim\Gin@nat@height>\textheight\textheight\else\Gin@nat@height\fi}
\makeatother
% Scale images if necessary, so that they will not overflow the page
% margins by default, and it is still possible to overwrite the defaults
% using explicit options in \includegraphics[width, height, ...]{}
\setkeys{Gin}{width=\maxwidth,height=\maxheight,keepaspectratio}
% Set default figure placement to htbp
\makeatletter
\def\fps@figure{htbp}
\makeatother
\usepackage[normalem]{ulem}
% Avoid problems with \sout in headers with hyperref
\pdfstringdefDisableCommands{\renewcommand{\sout}{}}
\setlength{\emergencystretch}{3em} % prevent overfull lines
\providecommand{\tightlist}{%
  \setlength{\itemsep}{0pt}\setlength{\parskip}{0pt}}
\setcounter{secnumdepth}{-\maxdimen} % remove section numbering
\ifluatex
  \usepackage{selnolig}  % disable illegal ligatures
\fi

\author{}
\date{\vspace{-2.5em}}

\begin{document}

\hypertarget{journal-week-01}{%
\section{Journal Week 01}\label{journal-week-01}}

\hypertarget{introducing-myself}{%
\subsubsection{Introducing Myself}\label{introducing-myself}}

My name is \textbf{Preston Lyons} and I'm using \sout{Github} RStudio,
to produce my first ever Journal entry. Though I don't have a Wikipedia
page, you can find out about the city of
\href{https://en.wikipedia.org/wiki/Preston,_Lancashire}{Preston} in
England, and \href{https://en.wikipedia.org/wiki/Lyon}{Lyon}\footnote{Note:
  just for clarity, my name is Lyons, not Lyon.} in France, in lieu of
my name not \emph{yet} being worth mentioning on Wikipedia.\footnote{Also
  note: I am neither English, nor French}

Another fact about myself is that I am doing research in the
\href{https://applieddevelopmentallab.com/}{Applied Developmental
Psychology Lab} which does a lot of work with children, and a quote I
quite like regarding children is that

\begin{quote}
Even a minor event in the life of a child is an event of that child's
world and thus a world event.
\end{quote}

by Gaston Bachelard.\footnote{I found a record of this quote
  \href{https://www.goodreads.com/quotes/471937-even-a-minor-event-in-the-life-of-a-child}{here}}

I've included a list of the help received for this Journal, including a
short summary of how they helped me:

\begin{enumerate}
\def\labelenumi{\arabic{enumi}.}
\tightlist
\item
  Joseph Bulbulia, whose enthusiasm for statistics encouraged me
\item
  Johannes Karl, who showed me the ropes with downloading, using, and
  figuring out how to use:
\end{enumerate}

\begin{itemize}
\tightlist
\item
  R
\item
  RStudio
\item
  Git
\item
  Github
\item
  GitKraken
\end{itemize}

\begin{enumerate}
\def\labelenumi{\arabic{enumi}.}
\setcounter{enumi}{2}
\tightlist
\item
  Some online glossary's and tutorials:
\end{enumerate}

\begin{itemize}
\tightlist
\item
  \href{https://commonmark.org/help/tutorial/index.html}{Commonmark}; -
  Danielle Navarro's
  \href{https://slides.djnavarro.net/starting-rmarkdown/\#1}{RMarkdown
  Tutorial}
\item
  \href{https://rmarkdown.rstudio.com/authoring_pandoc_markdown.html\#Pandoc_Markdown}{RMarkdown's}
  information
\item
  \href{https://bookdown.org/yihui/rmarkdown-cookbook/bibliography.html?fbclid=IwAR2UZgrVmVVolAr08TxKGeAEd-dxcb0gJHBgI477E48ZAuR7ZHpKRr9g-js}{Bookdown}
\item
  The \href{https://go-bayes.github.io/psych-447/about.htm}{PSYC447
  course materials page} authored by @Joseph Bulbulia
\end{itemize}

\begin{enumerate}
\def\labelenumi{\arabic{enumi}.}
\setcounter{enumi}{3}
\tightlist
\item
  I asked a classmate, Cam Taylor, for help with the bibliography
  section - and offered help if he was stuck in any area too.
\end{enumerate}

My impressions of the overall process is that it's clunky, it will lead
to teething issues, but will only get easier from here, and the end
product is going to be worth it. Johannes spent a lot of time during the
workshop getting Git and R to talk, and since he solved those problems,
I've been able to use RStudio without much hassle. I've needed to search
things up regularly when doing this journal, but am already starting to
retain information, and save, commit, push, and knit my work more and
more automatically. It's exciting to see how far I've already come, and
am looking forward to the challenges I anticipate with incorporating
statistics to this foundation. Unfortunately, I've not yet solved how to
do a bibliographic citation, despite Cam's assitance.

\hypertarget{attempted-bibliographic-citation}{%
\subsection{Attempted Bibliographic
Citation}\label{attempted-bibliographic-citation}}

bibliography: References.bib

references.bib

@Manual

@Joseph Bulbulia

\begin{center}\rule{0.5\linewidth}{0.5pt}\end{center}

\end{document}
